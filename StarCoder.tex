\input{BeamerHeader.tex}

\title{StarCoder: May the Source be With You!}
\author{Ramaseshan RAmachandran}
\date{\today}

\begin{document}

\begin{frame}
  \maketitle
\end{frame}

\begin{frame}{Introduction}
  \begin{itemize}
    \item StarCoder\cite{li2023starcodersourceyou}, a large language model, is designed for code generation.
    \item It aims to boost developer productivity with powerful code completion and generation.
    \item The model is trained on a vast dataset of source code from various languages.
    \item It emphasizes open access and transparency, fostering community collaboration.
  \end{itemize}
\end{frame}

\begin{frame}{Key Features}
  \begin{itemize}
    \item \textbf{Multilingual Support:} Trained on a wide range of programming languages.
    \item \textbf{Contextual Understanding:} Generates accurate and relevant code by leveraging context (code snippets).
    \item \textbf{Fill-in-the-Middle (FiM) Capabilities:} Allows code completion within existing code blocks.
    \item \textbf{Open Access:} Weights and code are released for public use and research.
    \item \textbf{Large Context Window:} Ability to process larger code snippets for better understanding.
    \begin{itemize}
    \item Utilizes a context window of 16,384 tokens
    \item Employs a sliding window attention of 4,096 tokens
    \item Allows the model to weigh the importance of each token in the input sequence.
    \end{itemize}
  \end{itemize}
\end{frame}

\begin{frame}{Training Data}
  \begin{itemize}
    \item The Stack dataset contains 1 trillion tokens
    \item 80+ programming languages
       \item Data preprocessing and filtering done - all symbols $=, <, >$, etc. are considered as a token .
    \item Emphasizes responsible data usage and transparency.
  \end{itemize}
\end{frame}
\begin{frame}{Training}
\item Fine-Tuning -  fine-tuned on 35B Python tokens
\end{frame}

\begin{frame}{Data selection - Section 3 - I}
\begin{itemize}
\item Programming Language Selection Criteria: Languages with over 500 MB of data, top 50 ranking on GitHub 2.0 or TIOBE Index, and dialects of selected languages.
\item Data exclusion criteria include configuration languages, unsupported languages, and data formats with limited volume.
\item Visual inspection of randomly selected files ensures high-quality code and filters applications based on file extensions.
\item Data filtering process involves XML, alpha, HTML, JSON, and YAML filters to remove irrelevant files.
\item Implemented to exclude XML files, especially those with extensions like .sld, based on the presence of “<?xml version=“ within the first 100 characters.
\item Alpha Filter identifies and removes files with a low percentage of alphabetic characters, especially in data files like MATLAB.

\end{itemize}
\end{frame}

\begin{frame}{Data selection - Section 3 - II}
\begin{itemize}
\item Converted Jupyter notebooks to scripts using Jupytext, identifying programming languages from metadata and guessing for unknown notebooks.
\item Filtered and merged Markdown and code blocks in Jupyter notebooks to create a dataset of consecutive code-text pairs.
\item Data source: 1.045 million structured Jupyter notebooks and natural language conversations from GitHub issues and pull requests.
\item Data preprocessing removed auto-generated text, short messages, long comments, and bot-generated content.
\item Conversations with two or more users or single-user conversations with less than 7,000 characters are included.
\item Data source: Stack dataset, Git commits, and GitHub issues.
\item Filtered out non-English issues, repositories with opted-out users, and deduplicated source code files.
\item Data anonymization replaced usernames in conversations with participant counters - ip address, password, keys, id, email, user name, etc.
\end{itemize}
\end{frame}

\begin{frame}{Architecture}
\begin{itemize}
\item Core Architecture: Decoder-Only Transformer - focus on predicting the next token in a sequence
\item Decoder-only-models: Process the input sequence and generate the output sequence in a single pass
\item Performance: The fine-tuning enhances capabilities, particularly in Python-specific tasks, improving overall code generation accuracy.
\end{itemize}
\begin{tabular}{l|l|l}
\hline
Hyperparameter &SantaCoder&\\
\hline
Hidden size:&2048&6144\\
Intermediate size& 8192&24576\\
Max. position embeddings& 2048&8192\\
Num. of attention heads& 16&48\\
Num. of hidden layers &24 &40\\
Attention &Multi-query&Multi-query\\
Num. of parameters&$\approx$1.1B &$\approx$5.5B\\
\hline
\end{tabular}
\end{frame}

\begin{frame}{Key Architectural Features}
\begin{itemize}
\item Multi-query attention, an optimization technique, improves the efficiency of the attention mechanism in SantaCoder.
\item This technique reduces memory footprint and speeds inference, enabling longer sequence handling.
\item Fill-in-the-Middle (FiM) is crucial to SantaCoder’s architecture. It lets the model generate code sequentially and fill missing code segments in existing blocks. This is valuable for code editing and refactoring.
\item Optimized for Code Tasks:
The architecture is optimized for code-related tasks, considering source code’s structured nature and syntax importance.
\end{itemize}
\end{frame}

\begin{frame}{Hardware setup}
\begin{itemize}
\item Trained on a GPU cluster with 512 A100 80 GB GPUs distributed across 64 nodes.
\item Model parallelism used a 3D-parallel layout with tensor and pipeline parallelism, needing 16 GPUs per replica.
\item Utilized 32-fold data parallelism with a micro-batch size of 1, accumulating for 16 steps to achieve a global batch size of 512.

\end{itemize}
\end{frame}
\begin{frame}{Impact and Applications}
    \begin{itemize}
        \item Code completion and generation tools.
        \item Assisting in code review and debugging.
        \item Automating repetitive coding tasks.
        \item Educational tool for learning programming.
        \item Research platform for further advancements in code LLMs.
    \end{itemize}
\end{frame}

\begin{frame}{Conclusion}
    \begin{itemize}
        \item StarCoder represents a significant step towards powerful and accessible code generation models.
        \item Its open nature promotes collaboration and innovation in the field.
        \item "May the source be with you!" emphasizes the importance of code accessibility and community.
    \end{itemize}
\end{frame}

%\setlength\pdfpagewidth{16cm}%
%\setlength\pdfpageheight{16cm}%
\begin{frame}[allowframebreaks]{References}
\begin{thebibliography}{99}
	\small
	\printbibliography
\end{thebibliography}
\end{frame}

\end{document}