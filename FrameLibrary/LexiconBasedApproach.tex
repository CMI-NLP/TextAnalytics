\begin{frame}{}
\Huge {Lexicon based Approach}
\end{frame}
\begin{frame}{What is a Lexicon?}
	\begin{itemize}
		\item A \textbf{lexicon} is a collection of words or vocabulary items in a language.
		\item It includes information about meanings, forms, and usage.
		\item Different domains (linguistics, NLP) may define lexicons slightly differently.
	\end{itemize}
\end{frame}

% Slide 2: Key Aspects
\begin{frame}{Key Aspects of a Lexicon}
	\begin{itemize}
		\item \textbf{Linguistics}:
		\begin{itemize}
			\item Entire set of words and phrases in a language or a person's mental vocabulary.
			\item Includes part of speech, pronunciation, and definitions.
		\end{itemize}
		\item \textbf{Computational Lexicon}:
		\begin{itemize}
			\item Structured resource for NLP containing words and their linguistic properties.
			\item May include synonyms, antonyms, semantic relationships, and grammar rules.
		\end{itemize}
	\end{itemize}
\end{frame}

% Slide 3: Lexicon vs. Dictionary
\begin{frame}{Lexicon vs. Dictionary}
	\begin{itemize}
		\item Both list words and meanings, but a lexicon is broader in scope.
		\item Lexicons focus on linguistic and semantic relationships.
		\item Dictionaries often provide formal definitions and examples.
	\end{itemize}
\end{frame}

% Slide 4: Examples of Lexicons
\begin{frame}{Examples of Lexicons}
	\begin{itemize}
		\item \textbf{General Purpose}:
		\begin{itemize}
			\item WordNet: Organizes words into synsets and provides relationships like hypernyms.
		\end{itemize}
		\item \textbf{Domain-Specific}:
		\begin{itemize}
			\item Medical or legal lexicons tailored to specialized fields.
		\end{itemize}
		\item \textbf{Multilingual Lexicons}:
		\begin{itemize}
			\item BabelNet: Bridges words across languages.
		\end{itemize}
	\end{itemize}
\end{frame}

% Slide 5: Applications of a Lexicon
\begin{frame}{Applications of a Lexicon}
	\begin{itemize}
		\item \textbf{Natural Language Processing}:
		\begin{itemize}
			\item Tokenization and word segmentation.
			\item Named Entity Recognition (NER).
			\item Sentiment analysis using sentiment lexicons.
		\end{itemize}
		\item \textbf{Language Learning}:
		\begin{itemize}
			\item Provides learners with vocabulary and usage examples.
		\end{itemize}
		\item \textbf{Search Engines and Information Retrieval}:
		\begin{itemize}
			\item Expands user queries with synonyms and related terms.
		\end{itemize}
	\end{itemize}
\end{frame}

% Slide 6: Conclusion
\begin{frame}{Conclusion}
	\begin{itemize}
		\item A lexicon is essential for understanding and processing language.
		\item It serves as the backbone of both human cognition and computational systems.
		\item Plays a critical role in NLP, linguistics, and language learning.
	\end{itemize}
\end{frame}

\begin{frame}{What is Lemmatization?}
	\begin{itemize}
		\item Lemmatization is the process of reducing words to their base or dictionary form, known as the \textbf{lemma}.
		\item It considers the context and performs morphological analysis to ensure the resulting word is meaningful.
	\end{itemize}
\end{frame}


\begin{frame}{Examples of Lemmatization}
	\begin{table}[]
		\centering
		\begin{tabular}{|c|c|}
			\hline
			\textbf{Word}  & \textbf{Lemma} \\
			\hline
			running        & run            \\
			studies        & study          \\
			children       & child          \\
			better         & good           \\
			\hline
		\end{tabular}
	\end{table}
\end{frame}

\begin{frame}{How Lemmatization Works}
	\begin{enumerate}
		\item \textbf{Morphological Analysis}: Analyzes the structure of the word.
		\item \textbf{POS Tagging}: Identifies the word's part of speech to determine the correct lemma.
		\item \textbf{Dictionary Lookup}: Ensures the lemma is an actual word.
	\end{enumerate}
\end{frame}


\begin{frame}{Tools for Lemmatization}
	\begin{itemize}
		\item \textbf{Python Libraries}:
		\begin{itemize}
			\item \textbf{NLTK}: Provides the \texttt{WordNetLemmatizer}.
			\item \textbf{spaCy}: Efficient lemmatization in its pipeline.
			\item \textbf{Stanford CoreNLP}: Robust lemmatization tool.
		\end{itemize}
	\end{itemize}
\end{frame}


\begin{frame}{Applications of Lemmatization}
	\begin{itemize}
		\item Preprocessing text for machine learning and NLP tasks.
		\item Enhancing search engine accuracy by normalizing queries.
		\item Useful in sentiment analysis, topic modeling, and text classification.
	\end{itemize}
\end{frame}

\begin{frame}{Lexicon-Based approach}

\begin{itemize}
\item [] Lemmatize all sentences
\item [] Let $\mathsf{L}$ represent the sentiment lexicon pairs, the word and its polarity
\item[] $Positives = 0, \, Negatives =0$
\item[] Test Sentence $S =\{w_1,w_2,w_3...w_n\}$
\item[] For $w$ in $S$
\begin{itemize}
\item []   If $w$ found in $\mathsf{L}$ and $w == positive$, then
	\item[] \quad\quad$Positives = Positives +$1
     \item[] Else
	\item[] \quad\quad$Negatives = Negatives +1$
\end{itemize}
\item[] If $Positives > Negatives$ then
     \item[] \quad\quad Return $positive$
    \item [] Else
    \item[] \quad\quad Return $negative$
\end{itemize}

\end{frame}