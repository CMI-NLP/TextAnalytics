\input{BeamerHeader.tex}
\title{Text Analytics}
\author{Ramaseshan Ramachandran}
\begin{document}

\frame{\titlepage}

\section{Introduction to Text Analytics}
\begin{frame}{What is Text Analytics?}
\begin{itemize}
    \item Converts unstructured text into insights
    \item Extracts patterns and trends
    \item Uses NLP, machine learning, and statistics
    \item Analyzes data from diverse sources
\end{itemize}
\end{frame}

\begin{frame}{Importance of Text Analytics}
\begin{itemize}
    \item Unstructured text data is vast
    \item Text data is messy and variable
    \item Makes data measurable and valuable
\end{itemize}
\end{frame}

\begin{frame}{Benefits of Text Analytics}
\begin{itemize}
    \item Understand customers and societal trends
    \item Enhance decision-making and planning
    \item Drive engagement and innovation
\end{itemize}
\end{frame}

\section{Key Activities in Text Analytics}
\begin{frame}{Information Retrieval (IR)}

\begin{multicols}{2}
\begin{itemize}
    \item Searches documents and metadata
    \item Extracts relevant data to queries
    \item Applies to large text collections
\end{itemize}
\vfill\null \columnbreak
\begin{itemize}
    \item Boolean Retrieval Model
    \item Vector Space Model (TF-IDF)
    \item Probabilistic Retrieval Model
    \item Latent Semantic Indexing (LSI)
    \item BM25 Algorithm
\end{itemize}

\end{multicols}
\end{frame}

\begin{frame}{Text Classification}
\begin{multicols}{2}
\begin{itemize}
    \item Categorizes text into predefined labels
    \item Applications: spam detection, sentiment analysis
    \item Automates document organization
\end{itemize}
\vfill\null\columnbreak
\begin{itemize}
    \item Naive Bayes Classifier
    \item Support Vector Machines (SVM)
    \item Logistic Regression
    \item Decision Trees and Random Forests
    \item Deep Learning (CNNs, RNNs, Transformers)
\end{itemize}
\end{multicols}
\end{frame}

\begin{frame}{Clustering}
\begin{multicols}{2}
\begin{itemize}
    \item Groups similar texts together
    \item No predefined labels required
    \item Useful for exploratory analysis
\end{itemize}

\vfill\null\columnbreak
\begin{itemize}
    \item K-Means Clustering
    \item Hierarchical Clustering
    \item DBSCAN (Density-Based Clustering)
    \item Gaussian Mixture Models (GMMs)
    \item Spectral Clustering
\end{itemize}
\end{multicols}
\end{frame}

\begin{frame}{Sentiment Analysis}
\begin{multicols}{2}
\begin{itemize}
    \item Determines text’s emotional tone
    \item Classifies as positive, negative, or neutral
    \item Applications: marketing, feedback analysis
\end{itemize}
\vfill\null\columnbreak
\begin{itemize}
    \item Lexicon-Based Approaches
    \item Rule-Based Sentiment Analysis
    \item Machine Learning-Based Approaches
    \item Neural Networks (LSTMs, GRUs, BERT)
    \item Pretrained Models (RoBERTa, GPT)
\end{itemize}
\end{multicols}
\end{frame}

\begin{frame}{Entity Recognition}
\begin{multicols}{2}
\begin{itemize}
    \item Identifies entities in text (e.g., names)
    \item Categorizes into people, places, etc.
    \item Useful for automated content analysis
\end{itemize}
\vfill\null\columnbreak
\begin{itemize}
    \item Hidden Markov Models (HMMs)
    \item Conditional Random Fields (CRFs)
    \item Maximum Entropy Models
    \item Neural Networks (BiLSTM + CRF)
    \item Pretrained Models (SpaCy, Hugging Face)
\end{itemize}
\end{multicols}
\end{frame}

\begin{frame}{Topic Modeling}
\begin{multicols}{2}
\begin{itemize}
    \item Discovers themes in document collections
    \item Applications: summarization, recommendations
\end{itemize}
\vfill \null
\columnbreak
\begin{itemize}
\item[] {\larger \underline{\textit{\color{yellow}Algorithms}}}
    \item Latent Dirichlet Allocation (LDA)
    \item Non-Negative Matrix Factorization (NMF)
    \item Latent Semantic Analysis (LSA)
    \item Gibbs Sampling for LDA
    \item Neural Topic Models (ProdLDA, BERTopic)
\end{itemize}
\end{multicols}
\end{frame}
\section{Advanced Techniques}
\begin{frame}{Other Advanced Techniques}
\begin{itemize}
    \item Word Embedding Models (Word2Vec, GloVe, FastText)
    \item Sentence Embeddings (Sentence-BERT)
    \item Attention Mechanisms
    \item Transformers (BERT, GPT, T5)
    \item Text Summarization (Extractive and Abstractive)
\end{itemize}
\end{frame}

\section{Conclusion}
\begin{frame}{Conclusion}
\begin{itemize}
    \item Text analytics unlocks data potential
    \item Drives decisions and innovation
    \item Benefits industries like finance and healthcare
    \item A variety of algorithms drive text analytics
    \item Techniques range from statistical to neural
    \item Tailor solutions based on use case
\end{itemize}
\end{frame}
\begin{frame}{Word embedding cannot fight with others. Why}
\begin{center}
	\includegraphics[width=0.55\linewidth]{Images/TAJoke1}
\end{center}

\end{frame}


	% Slide 2: What is Scikit-Learn?
	\begin{frame}{What is Scikit-Learn?}
		\begin{itemize}
			\item {\color{yellow}Definition:} Scikit-learn is a Python library for machine learning, providing tools for:
			\begin{itemize}
				\item Classification
				\item Regression
				\item Clustering
				\item Dimensionality reduction
				\item Preprocessing and more
			\end{itemize}
			\item \textbf{Key Features:}
			\begin{itemize}
				\item Built on NumPy, SciPy, and matplotlib
				\item Simple and efficient tools for predictive data analysis
				\item Open source
			\end{itemize}
		\end{itemize}
	\end{frame}
	
	% Slide 3: Why Use Scikit-Learn in Text Analytics?
	\begin{frame}{Why Use Scikit-Learn in Text Analytics?}
		\begin{itemize}
			\item \textbf{Text Analytics Focus:}
			\begin{itemize}
				\item Natural Language Processing (NLP) tasks
				\item Feature extraction (e.g., bag-of-words, TF-IDF)
				\item Building predictive models
			\end{itemize}
			\item \textbf{Advantages:}
			\begin{itemize}
				\item Wide range of algorithms (SVMs, Naive Bayes, etc.)
				\item User-friendly API for rapid prototyping
				\item Extensive documentation and community support
			\end{itemize}
		\end{itemize}
	\end{frame}
	
	% Slide 4: Common Use Cases in Text Analytics
	\begin{frame}{Common Use Cases in Text Analytics}
		\begin{itemize}
			\item \textbf{Examples:}
			\begin{itemize}
				\item Spam detection
				\item Sentiment analysis
				\item Topic modeling
				\item Document classification
			\end{itemize}
			\item \textbf{Techniques:}
			\begin{itemize}
				\item Preprocessing: Tokenization, stemming, lemmatization
				\item Vectorization: TF-IDF or CountVectorizer
				\item Model training: Logistic regression, SVMs, etc.
			\end{itemize}
		\end{itemize}
	\end{frame}
	
	% Slide 5: Typical Workflow in Scikit-Learn
	\begin{frame}{Typical Workflow in Scikit-Learn}
		\begin{enumerate}
			\item \textbf{Data Preprocessing:}
			\begin{itemize}
				\item Cleaning text data (e.g., removing stop words)
				\item Vectorization (e.g., TF-IDF)
			\end{itemize}
			\item \textbf{Model Selection:}
			\begin{itemize}
				\item Choosing an algorithm (e.g., Naive Bayes)
			\end{itemize}
			\item \textbf{Model Training:}
			\begin{itemize}
				\item \texttt{model.fit(X\_train, y\_train)}
			\end{itemize}
			\item \textbf{Model Evaluation:}
			\begin{itemize}
				\item Metrics like accuracy, precision, recall
			\end{itemize}
			\item \textbf{Prediction:}
			\begin{itemize}
				\item \texttt{model.predict(X\_test)}
			\end{itemize}
		\end{enumerate}
	\end{frame}
	
	% Slide 6: Example: Text Classification Workflow
	\begin{frame}{Example: Text Classification Workflow}
		\textbf{Dataset:} Sentiment Analysis on Product Reviews
		\begin{enumerate}
			\item Load dataset
			\item Preprocess text (lowercase, remove punctuation, etc.)
			\item Vectorize with TF-IDF
			\item Train model (e.g., Logistic Regression)
			\item Evaluate using accuracy and F1-score
		\end{enumerate}
\end{frame}
		\begin{frame}[fragile]{Code Example:}
		\begin{lstlisting}
from sklearn.feature_extraction.text import TfidfVectorizer
from sklearn.model_selection import train_test_split
from sklearn.linear_model import LogisticRegression
from sklearn.metrics import accuracy_score

vectorizer = TfidfVectorizer()
X = vectorizer.fit_transform(text_data)
X_train, X_test, y_train, y_test = train_test_split(X, labels, test_size=0.2)
model = LogisticRegression()
model.fit(X_train, y_train)
predictions = model.predict(X_test)
print("Accuracy:", accuracy_score(y_test, predictions))
		\end{lstlisting}
	\end{frame}
	
	% Slide 7: Strengths and Limitations
	\begin{frame}{Strengths and Limitations}
		\begin{itemize}
			\item \textbf{Strengths:}
			\begin{itemize}
				\item Easy to use and integrate
				\item Extensive support for text-related tasks
				\item Scalability for moderate-sized datasets
			\end{itemize}
			\item \textbf{Limitations:}
			\begin{itemize}
				\item Not designed for deep learning
				\item Limited support for out-of-core learning
			\end{itemize}
		\end{itemize}
	\end{frame}
	
	% Slide 8: Resources to Learn More
	\begin{frame}{Resources to Learn More}
		\begin{itemize}
			\item Official Documentation: \url{https://scikit-learn.org}
			\item Tutorials:
			\begin{itemize}
				\item "Getting Started with Scikit-Learn" (Blog/Video)
				\item Kaggle courses on ML
			\end{itemize}
			\item Recommended Books:
			\begin{itemize}
				\item \textit{Python Machine Learning} by Sebastian Raschka
				\item \textit{Introduction to Machine Learning with Python} by Andreas Müller
			\end{itemize}
		\end{itemize}
	\end{frame}
	
	% Slide 9: Q&A
	\begin{frame}{Questions \& Answers}
		\centering
		\Large Questions?
	\end{frame}
	


\end{document}