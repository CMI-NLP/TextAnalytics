%\RequirePackage[2020-02-02]{latexrelease}
\documentclass[11pt,aspectratio=169, professionalfonts]{beamer}
%\usepackage[orientation=landscape,size=custom,width=16,height=9,scale=0.4]{beamerposter}
%\useoutertheme{infolines}
% or other languages

%\usepackage[latin9]{inputenc}


\usepackage{tasks}
\usepackage{multicol}
\usepackage{amsmath}
\usepackage{bookmark}
\usepackage{fnpct}
\usepackage{vwcol}
\usepackage{python}
\usepackage[backend=biber, style=numeric,sorting=none]{biblatex}
\usepackage{marvosym}
\usepackage{amsthm}
\usepackage[percent]{overpic}
\usepackage{svg}
\usepackage{multimedia}
\usepackage{rotating}
\usepackage{pdfpages}
\usepackage{tabulary}
\usepackage{vwcol}
\usepackage{caption}
\usepackage{tikz}
\usepackage{tikz-qtree}
\usepackage{pgfpages}
\usepackage{multirow}
\usepackage{wrapfig}
\usepackage{tcolorbox}
\usepackage{rotating}
\usepackage{lmodern}% http://ctan.org/pkg/lm
\usepackage{anyfontsize}
\usepackage{algorithm,algorithmic}
%\usepackage[ruled,vlined]{algorithm2e}
\usepackage{multicol}
\usepackage{mathptmx}
\usepackage{mathtools}
\usepackage{pgfplots}
\usepackage{pgfplotstable}
\usepackage{outlines}

\DeclareGraphicsExtensions{.pdf,.png}

\usepackage{amsmath}
\usepackage{amssymb}
\usepackage{pgf}
\usepackage{listings}
\usepackage{hhline}
\usepackage{hyperref}
\usepackage{multicol}
\usepackage{soul, xcolor}
\usepackage{graphicx}
\usepackage{vwcol}
\usepackage{color, colortbl}
\usepackage{tikz}
\usepackage{pgfplots}
\usepackage[absolute,overlay]{textpos}
%\usepackage{algpseudocode}
\usepackage{relsize}
\usepackage{venndiagram}
%\usepackage[framemethod=TikZ]{mdframed}
\usepackage{makeidx}
\usepackage{cleveref}
\usepackage{xifthen}
\usepackage{xparse}
%\usepackage{mdframed}
\usepackage{accents}
\usepackage{beton}
\usepackage{euler}
\usepackage{tikz}
\usepackage[T1]{fontenc}
\usepackage{transparent}
\usepackage{varwidth}
\usepackage{mwe} % provides example image
\usepackage{emoji}
\setemojifont{Apple Color Emoji}  % Optional

%\usepackage{enumitem}
\normalfont
\input{/Users/ram/Documents/GitHub/NLProc/Latex/ColorCodes.tex}
% Override palette coloring with secondary
\setbeamercolor{subsection in head/foot}{,fg=white}

\setbeamercolor{normal text}{fg=white,bg=black}

\setbeamercolor{structure}{fg=white}
\setbeamercolor{definition}{fg=white}

%\setbeamercolor{alerted text}{fg=red!85!black}

%\setbeamercolor{item projected}{use=item,fg=black,bg=item.fg!35}

\setbeamercolor{note page}{bg=white}
\setbeamercolor{title}{bg=black, fg=yellow}
\setbeamercolor{author}{bg=black, fg=white}
\setbeamercolor{subtitle}{bg=black, fg=white}
%\setbeamercolor{frametitle}{bg=black, fg=yellow}
\setbeamercolor{frametitle}{fg=orange,bg=black}
\setbeamertemplate{frametitle}{%
\usebeamerfont{frametitle} \MakeUppercase{\insertframetitle}%
\vphantom{g}% To avoid fluctuations per frame
%\hrule% Uncomment to see desired effect, without a full-width hrule
\par% <-- added
\vspace{-3mm}
{\color{white}\hrulefill}
}

\addtobeamertemplate{navigation symbols}{}{%
	\usebeamerfont{footline}%
	\usebeamercolor[fg=yellow]{footline}%
	\hspace{1em}%
	\insertframenumber/\inserttotalframenumber
}



\setbeamertemplate{footline}
{
	\leavevmode%
	\hbox{%
		\begin{beamercolorbox}[wd=.333333\paperwidth,ht=2.25ex,dp=1ex,center]{subsection in head/foot}%
			\usebeamerfont{subsection in head/foot}\insertsection
		\end{beamercolorbox}%
		\begin{beamercolorbox}[wd=.333333\paperwidth,ht=2.25ex,dp=1ex,center]{subsection in head/foot}%
			\usebeamerfont{title in head/foot}\insertshorttitle
		\end{beamercolorbox}%
		\begin{beamercolorbox}[wd=.333333\paperwidth,ht=2.25ex,dp=1ex,right]{section in head/foot}%
			\usebeamerfont{date in head/foot}\insertshortdate{}\hspace*{2em}
			\insertframenumber{} / \inserttotalframenumber\hspace*{2ex}
	\end{beamercolorbox}}%
	\vskip0pt%
}
\makeatother

\usetikzlibrary{chains,shapes}
\tikzstyle{MyText} = [text width=0.5cm,text centered]
\pgfplotsset{compat=1.11} %<------ Or use this one
\usetikzlibrary{decorations.text}
\setlength{\columnseprule}{0.2pt}
\def\columnseprulecolor{\color{yellow}}
\usepackage{nicefrac}
\usetikzlibrary{automata, shapes.geometric,circuits,positioning,shapes,arrows,fit,calc,decorations.pathmorphing,decorations}
% or whatever (possibly just delete it)


\usepackage[printwatermark=true]{xwatermark}



%\SetWatermarkText{\includegraphics{JCRLogo.png}}
%\setbeamertemplate{background}{
%	\tikz[overlay,remember picture]\node[opacity=0.04]at (current page.center){\includegraphics[width=4cm]{JCRLogo}};
%	}
\lstset{
backgroundcolor=\color{myblue},   % choose the background color; you must add \usepackage{color} or \usepackage{xcolor}; should come as last argument
basicstyle=\ttfamily\scriptsize,        % the size of the fonts that are used for the code
breakatwhitespace=false,         % sets if automatic breaks should only happen at whitespace
breaklines=false,                 % sets automatic line breaking
captionpos=b,                    % sets the caption-position to bottom
commentstyle=\color{green!25},    % comment style
deletekeywords={...},            % if you want to delete keywords from the given language
escapeinside={\%*}{*)},          % if you want to add LaTeX within your code
extendedchars=true,              % lets you use non-ASCII characters; for 8-bits encodings only, does not work with UTF-8
%frame=single,	                   % adds a frame around the code
keepspaces=true,                 % keeps spaces in text, useful for keeping indentation of code (possibly needs columns=flexible)
keywordstyle=\color{orange},       % keyword style
basicstyle=\small\ttfamily
morekeywords={*,...},            % if you want to add more keywords to the set
numbers=left,                    % where to put the line-numbers; possible values are (none, left, right)
numbersep=5pt,                   % how far the line-numbers are from the code
numberstyle=\tiny\color{cyan}, % the style that is used for the line-numbers
rulecolor=\color{white},         % if not set, the frame-color may be changed on line-breaks within not-black text (e.g. comments (green here))
showspaces=false,                % show spaces everywhere adding particular underscores; it overrides 'showstringspaces'
showstringspaces=false,          % underline spaces within strings only
showtabs=false,                  % show tabs within strings adding particular underscores
stepnumber=0,                    % the step between two line-numbers. If it's 1, each line will be numbered
stringstyle=\color{green},     % string literal style
tabsize=4,	                   % sets default tabsize to 2 spaces
%title=\lstname                   % show the filename of files included with \lstinputlisting; also try caption instead of title
belowskip=-5pt,
basicstyle=\ttfamily,
keywordstyle = \color{yellow},
language=Python,
}



\graphicspath{{Images/}}



\hypersetup{
colorlinks=true,
linkcolor=blue,
filecolor=magenta,
urlcolor=cyan,
}

\urlstyle{same}

\newsavebox\mybox
%\savebox\mybox{ \tikz[opacity=0.1]  \node {\includegraphics{JCRLogo}};}
\newwatermark*[
angle=0,
allpages=true,
scale=0.1,
xpos=-5.75,
ypos=0
]{\usebox\mybox}

%\AtBeginSection[]{
%	\begin{frame}
%
%		\centering
%		%\begin{beamercolorbox}[sep=8pt,center,shadow=false,rounded=true]{title}
%			\usebeamerfont{title}\insertsectionhead\par%
%		\end{beamercolorbox}
%
%	\end{frame}
%}



\tikzstyle{mybox} = [draw=white, fill=blue!75, very thick,
rectangle, rounded corners, inner sep=10pt, inner ysep=10pt]
\tikzstyle{fancytitle} =[draw=white,fill=blue!75, text=white, ellipse, very thick]

%\usecolortheme{owl}


\newcommand{\ffbox}[1]{%
	{% open a group for a local setting
		\setlength{\fboxsep}{-2\fboxrule}% the rule will be inside the box boundary
		\fbox{\hspace{1.2pt}\strut#1\hspace{1.2pt}}% print the box, with some padding at the left and right
	}% close the group
}

\tikzset{
	NNnode/.pic={
		\pgfmathsetmacro\RecH{2}
		\pgfmathsetmacro\RecW{\RecH/10}
		\coordinate (-ll) at (-\RecW/2,-\RecH/2);
		\coordinate (-ur) at (\RecW/2,\RecH/2);
		\coordinate (-lr) at (-ll-|-ur);
		\coordinate (-ul) at (-ll|--ur);
		\path (-ul) -- (-ur) coordinate[midway] (-north);
		\path (-ll) -- (-lr) coordinate[midway] (-south);
		\path (-ll) -- (-ul) coordinate[midway] (-west);
		\path (-ur) -- (-lr) coordinate[midway] (-east);

		\begin{scope}[shift={(-\RecW/2,-\RecH/2)}]
			\draw (-ll) rectangle (-ur);
			\foreach \y in {0.05,0.5,0.75,0.85,0.95}
			\draw (0.5*\RecW,\RecH*\y) circle[radius=0.3*\RecW];
			\foreach \y in {0.275,0.625} {
				\fill (\RecW*0.4,\y*\RecH-0.1*\RecW) rectangle (0.6*\RecW,\y*\RecH-0.3*\RecW);
				\fill (\RecW*0.4,\y*\RecH+0.1*\RecW) rectangle (0.6*\RecW,\y*\RecH+0.3*\RecW);
			}
		\end{scope}
	}
}
\newenvironment{WrapText}[1][c]
{\wrapfigure{}{0.35\textwidth}\tcolorbox}
{\endtcolorbox\endwrapfigure}


\tikzstyle{decision} = [diamond, draw, text width=4.5em, text badly centered, node distance=3.5cm, inner sep=0pt]
\tikzstyle{block} = [rectangle, draw, text width=4em, text centered, rounded corners, minimum height=4em, node distance=0.5cm,minimum height=5em]
\tikzstyle{hidden} = [rectangle, draw, text width=0.5em, text centered, rounded corners, minimum height=2.5em, node distance=0.6cm,minimum height=2.5em,fill,red,opacity=0.85]
\tikzstyle{cloud} = [draw, ellipse, node distance=3.5cm, minimum height=2em]
\tikzstyle{line} = [draw, -latex']

\tikzstyle{vecArrow} = [thick, decoration={markings,mark=at position
	1 with {\arrow[semithick]{open triangle 60}}},
double distance=1.4pt, shorten >= 5.5pt,
preaction = {decorate},
postaction = {draw,line width=1.4pt, white,shorten >= 4.5pt}]
\tikzstyle{innerWhite} = [semithick, white,line width=1.4pt, shorten >= 4.5pt]

\DeclareMathOperator*{\argmin}{\arg\min}
\DeclareMathOperator*{\argmax}{\arg\max}

\tikzstyle{MyText} = [text width=0.5cm,text centered]


\newcommand\irregularline[2]{%
	let \n1 = {rand*(#1)} in
	+(0,\n1)
	\foreach \a in {0.1,0.2,...,#2}{
		let \n1 = {rand*(#1)} in
		-- +(\a,\n1)
	}
}

\usefonttheme{professionalfonts}
\setbeamercolor{frametitle}{fg=orange}
\setbeamerfont{frametitle}{size=\large}
\beamertemplatenavigationsymbolsempty


\newcommand*\circled[1]{\tikz[baseline=(char.base)]{
		\node[shape=circle,draw,inner sep=1pt] (char) {#1};}}

\setbeamertemplate{section in toc}{%
	\usebeamercolor[fg]{enumerate item}%
	\makebox[2em][l]{\circled{\inserttocsectionnumber}}%
	\parbox[t]{\dimexpr\linewidth-2em}{\inserttocsection}%
}
\setbeamercolor{item projected}{bg=orange}
\setbeamercolor*{item}{fg=orange}


%\bibliography{/Users/ram/Documents/GitHub/NLProc/Latex/Bib/NLP.bib}

%\setbeamertemplate{background}{ \tikz[overlay,remember picture]\node[opacity=0.05]at (current page.center){\includegraphics[width=4cm]{RamLogo.png}};}

%\usepackage[hyphenbreaks]{breakurl} % benötigt für das Brechen von URLs in Literaturreferenzen, hyphenbreaks auch bei links, die über eine Seite gehen (mit hyphenation).
\usepackage{xcolor}
\definecolor{c1}{rgb}{1,1,0} % blue
\definecolor{c2}{rgb}{1,1,1} % light blue
\definecolor{c3}{rgb}{1,1,0} % red blue
\hypersetup{
	linkcolor= {c1}, % internal links
	citecolor={c2}, % citations
	urlcolor={c3} % external links/urls
}

\setbeamercolor{background canvas}{bg=myblue}

\setbeamercolor{note page}{bg=white}
\setbeamercolor{title}{bg=myblue, fg=gold}
\setbeamercolor{author}{bg=myblue, fg=white}
\setbeamercolor{subtitle}{bg=myblue, fg=white}
\setbeamercolor{frametitle}{fg=beamer_title_color,bg=myblue}
\addbibresource{/Users/ram/Documents/GitHub/NLProc/Latex/Bib/NLP.bib}

\logo{
	\begin{tikzpicture}[remember picture,overlay]
        \node[above right,inner sep=0pt] at (current page.south west) {
           \ifnum\thepage>1 \includegraphics[width=2.0cm]{corner_image_2.png}\fi
        };
      \end{tikzpicture}
      }

\newcommand{\clrtxt}[2]{\textcolor{#1}{#2}}
\date{}

