
\documentclass{article}
% \usepackage{fontspec}
\usepackage[a4paper]{geometry}
\usepackage{hyperref}
\begin{document}
\section{ABSA in Indian Languages}
Aspect-Based Sentiment Analysis (ABSA) has experienced a surge in research in Indian languages, particularly addressing challenges such as resource scarcity and linguistic complexity. This section provides a comprehensive overview of key developments in the field.
\section{Hindi ABSA}
\begin{itemize}
\item Pioneering Work: First comprehensive ABSA framework with 5,417 annotated reviews across 12 domains, addressing \item aspect term extraction (41.04\% F1-score using CRF) and **sentiment classification** (54.05\% accuracy with SVM) [4][7][8].
\item Advanced Models: Ensembled mBERT models achieved state-of-the-art results for aspect category detection and polarity classification, demonstrating transfer learning’s effectiveness in low-resource settings [1][15].

\item Domain Coverage: Includes restaurants, electronics, and hospitality. Annotated datasets for aspect categories like Price, Food, and Service [9][16].

\end{itemize}

\section{Bengali ABSA}
\begin{itemize}
\item BERT Adoption: First use of BERT for ABSA, outperforming traditional models (SVM, CNN) on datasets from social media and news portals [2].
\item Aspect Annotation**: Manually labeled datasets across five aspects (*Technology*, *Corruption*) and three sentiment polarities [2].
\end{itemize}

\section{Telugu ABSA}
\begin{itemize}
\item Benchmark Dataset: Annotated corpus for three tasks: aspect term extraction, polarity classification, and categorization, validated via BiLSTM and CNN models [6][29].
\item Multi-Domain Focus: Includes reviews on electronics, books, and movies [6].

\end{itemize}
\section{Odia ABSA}
\begin{itemize}
\item First Benchmark**: Newly created dataset for aspect term extraction and polarity classification across seven domains (e.g., *Handloom Sarees*, *Odia Movies*), achieving 77.20\% accuracy with IndicBERT [3].
\item Data Augmentation**: Combined back-translation and T5 paraphrase generation to overcome data scarcity [3].
\end{itemize}

\section{Malayalam and Others}
\begin{itemize}
\item Emerging Work**: Preliminary studies using machine learning, though resources remain limited [11][14].
\item Cross-Language Surveys**: Highlight progress in Hindi, Bengali, Telugu, Urdu, and Malayalam while identifying gaps in Kannada, Nepali, and other languages [5][14].

\end{itemize}
\section{Challenges and Trends}
\begin{itemize}
\item Resource Limitations**: Most studies rely on small, manually annotated datasets (e.g., 1,000–5,000 sentences) [3][6].
\item Model Innovations: Shift from CRF/SVM (2016–2020) to transformer-based models like mBERT and XLM-R (2021–2025) [1][3][15].
\item Domain Generalization: Efforts to align datasets with international benchmarks (e.g., SemEval) for better comparability [3][6].

\end{itemize}
While ABSA research in Indian languages is advancing, languages like Odia and Malayalam remain under-resourced compared to Hindi and Bengali. Recent work emphasizes cross-lingual transfer learning and synthetic data generation to bridge this gap [3][5].

\section{Useful resources}
\begin{enumerate}
\item \url{https://www.mdpi.com/2079-9292/10/21/2641}{https://www.mdpi.com/2079-9292/10/21/2641}
\item \url {https://thesai.org/Downloads/Volume13No12/Paper_112-Aspect_based_Sentiment_Analysis_for_Bengali_Text.pdf}
\item \url{https://aclanthology.org/2025.coling-main.391.pdf}
\item \url{https://www.cse.iitb.ac.in/~pb/papers/lrec16-sentiment-resource.pdf}
\item \url{https://www.cfilt.iitb.ac.in/resources/surveys/2022/kunal_CrossLingualABSA_survey_2022.pdf}
\item \url{https://aclanthology.org/2020.lrec-1.617/}
\item \url{https://iitp.ac.in/~shad.pcs15/data/Tutorial-SA-Hindi-GIAN.pdf}
\item \url{https://www.cse.iitb.ac.in/~pb/papers/cicling16-aspect-based-sa.pdf}
\item \url{https://aclanthology.org/L16-1429/}
\item \url{https://www.researchgate.net/publication/372289987_ASPECT-BASED_SENTIMENT_ANALYSIS_A_COMPREHENSIVE_SURVEY_OF_TECHNIQUES_AND_APPLICATIONS}
\item \url{https://ieeexplore.ieee.org/document/10143021/}
\item \url{https://web2py.iiit.ac.in/research_centres/publications/download/mastersthesis.pdf.b42c721e7c36be1e.5961736877616e74685f5468657369735f66696e616c2e706466.pdf}
\item \url{https://arxiv.org/html/2311.10777v4}
\item \url{https://ieeexplore.ieee.org/document/10392226/}
\item \url{https://www.researchgate.net/publication/355871180_Aspect-Based_Sentiment_Analysis_in_Hindi_Language_by_Ensembling_Pre-Trained_mBERT_Models}
\item \url{https://www.researchgate.net/publication/323875540_Aspect_Based_Sentiment_Analysis_Category_Detection_and_Sentiment_Classification_for_Hindi}
\item \url{https://ieeexplore.ieee.org/document/9402365/}
\item \url{https://dl.acm.org/doi/10.1145/3485243}
\item \url{https://www.researchgate.net/publication/379119554_Rule-Based_Approach_in_Aspect-Based_Sentiment_Analysis_of_Hindi_Text}
\item \url{https://paperswithcode.com/datasets?q=Hostility+Detection+Dataset+in+Hindi&lang=telugu}
\item \url{https://www.researchgate.net/figure/Flow-of-Design-for-Odia-SentiWordNet_fig1_322582732}
\item \url{https://dl.acm.org/toc/tallip/current}
\item \url{https://www.researchgate.net/publication/329667038_A_journey_of_Indian_languages_over_Sentiment_analysis_a_systematic_review}
%\item \url{https://www.linkedin.com/posts/absa-cib_africaexpertise-activity-7213892896402337793-WjNp}
\item \url{https://www.ijnrd.org/viewpaperforall.php?paper=IJNRD2303129}
\item \url{https://iitp.ac.in/~shad.pcs15/data/Tutorial-SA-Hindi-GIAN.pdf}
\item \url{https://aclanthology.org/2020.lrec-1.617/}
\item \url{https://www.mdpi.com/2079-9292/10/21/2641}
\end{enumerate}
\end{document}
