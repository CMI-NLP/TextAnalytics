    \documentclass[10pt,a4paper]{article}
\usepackage{geometry}
\newgeometry{vmargin={25mm}, hmargin={10mm,10mm}}
\usepackage[dvipsnames,svgnames]{xcolor}
\usepackage{amsmath}
\usepackage{amssymb}
\usepackage{listings}
\usepackage{tasks}
\usepackage{enumitem}
\usepackage[normalem]{ulem}
\usepackage{fancyhdr}
\usepackage{graphicx,amsmath,amssymb}
\usepackage[bookmarks,hypertexnames=false,debug,linktocpage=true,hidelinks]{hyperref}
\usepackage{longtable, amsmath, amsfonts, wrapfig, graphicx, arydshln,comment,stmaryrd,tikz,multicol,mathrsfs,mathtools,nicematrix,colortbl,lmodern,tikzpagenodes,enumitem,tasks,fontsize, emoji}
\hypersetup{
    colorlinks=true,
    linkcolor=red,
    filecolor=red,
    pdfborderstyle={/S/U/W 1}
}

\pagestyle{fancy}
\setlist[1]{itemsep=1pt, parsep=1pt}

\lstset{
    basicstyle=\ttfamily\scriptsize,        % the size of the fonts that are used for the code
    breakatwhitespace=false,         % sets if automatic breaks should only happen at whitespace
    breaklines=true,                 % sets automatic line breaking
    captionpos=b,                    % sets the caption-position to bottom
    commentstyle=\color{DarkGreen},    % comment style
    deletekeywords={...},            % if you want to delete keywords from the given language
    escapeinside={\%*}{*)},          % if you want to add LaTeX within your code
    extendedchars=true,              % lets you use non-ASCII characters; for 8-bits encodings only, does not work with UTF-8
    frame=single,	                   % adds a frame around the code
    keepspaces=true,                 % keeps spaces in text, useful for keeping indentation of code (possibly needs columns=flexible)
    keywordstyle=\color{orange},       % keyword style
    morekeywords={*,...},            % if you want to add more keywords to the set
    numbers=left,                    % where to put the line-numbers; possible values are (none, left, right)
    numbersep=5pt,                   % how far the line-numbers are from the code
    numberstyle=\tiny\color{Blue}, % the style that is used for the line-numbers
    rulecolor=\color{white},         % if not set, the frame-color may be changed on line-breaks within not-black text (e.g. comments (green here))
    showspaces=false,                % show spaces everywhere adding particular underscores; it overrides 'showstringspaces'
    showstringspaces=false,          % underline spaces within strings only
    showtabs=false,                  % show tabs within strings adding particular underscores
    stepnumber=0,                    % the step between two line-numbers. If it's 1, each line will be numbered
    stringstyle=\color{ForestGreen},     % string literal style
    tabsize=4,	                   % sets default tabsize to 2 spaces
    %title=\lstname                   % show the filename of files included with \lstinputlisting; also try caption instead of title
    belowskip=0pt,
    language=Python,
}


    %opening
    \title{Assignment 3}
    \date{}
    \begin{document}
        %\fancyhead[c]{Assignment 1}
        \fancyhead[R]{\textbf{Assignment 3}}
        %	\fancyfoot[c]{Use extra sheets if needed, clearly marking the question number against each answer}
        %\fancycenter[10pt]{\textbf{Assignment 1}}{Advanced Programming}{\textbf{Submit by 05-Feb-2024}}
        \section{Instructions}
        Please note the following guidelines for submitting your assignments:
        \begin{enumerate}
            \item Use \href{https://colab.research.google.com}{Colab}  to submit your assignments.
            \item You may use  any machine learning library (pyTorch, Tensorflow, etc.) to implement this assignment
            \item Share the final version of your assignments with the following email ID:
            \begin{enumerate}
                \item  ramaseshan.ta@gmail.com
            \end{enumerate}

            \item I will NOT run/change your Python notebook.
            \item Naming Conventions for the assignments:
            \begin{enumerate}
                \item The first part of the filename should be your First name.
                \item The second part of the filename should be your roll number.
                \item The third part of your assignment should be Assignment0X, where X is the assignment number.
                \item[] \textbf{Example:} \verb|SriramBMC202204_Assignment01.ipynb|
            \end{enumerate}
            \item Make sure that all the results are available when you share them.
            \item \textbf{Optional}: You may use Github to store your versions of the assignment. Advantages - You will never lose your code if you check-in the code into the GitHub repository after changes regularly.
        \end{enumerate}
        \pagebreak
            \section{Assignment 3 (A3) - Parts of Speech Tagging with Recurrent Neural Networks (RNNs)}


\subsection{Objective}
In this assignment, you will build a knowledge graph using similar words using the word vectors (common-crawl, anyone) from these pre-trained word vectors. Gensim provides options to load the GloVe word vectors in word2vec format.
\end{document}
