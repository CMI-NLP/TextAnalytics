    \input{./PSHeader.tex}


    %opening
    \title{Assignment 3}
    \date{}
    \begin{document}
        %\fancyhead[c]{Assignment 1}
        \fancyhead[R]{\textbf{Assignment 3}}
        %	\fancyfoot[c]{Use extra sheets if needed, clearly marking the question number against each answer}
        %\fancycenter[10pt]{\textbf{Assignment 1}}{Advanced Programming}{\textbf{Submit by 05-Feb-2024}}
        \section{Instructions}
        Please note the following guidelines for submitting your assignments:
        \begin{enumerate}
            \item Use \href{https://colab.research.google.com}{Colab}  to submit your assignments.
            \item You may use  any machine learning library (pyTorch, Tensorflow, etc.) to implement this assignment
            \item Share the final version of your assignments with the following email ID:
            \begin{enumerate}
                \item  ramaseshan.ta@gmail.com
            \end{enumerate}

            \item I will NOT run/change your Python notebook.
            \item Naming Conventions for the assignments:
            \begin{enumerate}
                \item The first part of the filename should be your First name.
                \item The second part of the filename should be your roll number.
                \item The third part of your assignment should be Assignment0X, where X is the assignment number.
                \item[] \textbf{Example:} \verb|SriramBMC202204_Assignment01.ipynb|
            \end{enumerate}
            \item Make sure that all the results are available when you share them.
            \item \textbf{Optional}: You may use Github to store your versions of the assignment. Advantages - You will never lose your code if you check-in the code into the GitHub repository after changes regularly.
        \end{enumerate}
        \pagebreak
            \section{Assignment 3 (A3) - Parts of Speech Tagging with Recurrent Neural Networks (RNNs)}


\subsection{Objective}
In this assignment, you will build a knowledge graph using similar words using the word vectors (common-crawl, anyone) from these pre-trained word vectors. Gensim provides options to load the GloVe word vectors in word2vec format.
\end{document}
